\documentclass[10pt,a4paper]{article}
\usepackage{hyperref}
\usepackage{cleveref}
\hypersetup{hypertexnames = false, bookmarksdepth = 2, bookmarksopen = true, colorlinks, linkcolor = black, citecolor = black, urlcolor = black, pdfstartview={XYZ null null 1}}

\usepackage{amsfonts}
\usepackage{amsmath}
\usepackage{amsthm}
\usepackage[greek, dutch]{babel}
\usepackage{enumitem}
\usepackage{mathtools}
\usepackage{thmtools}
\usepackage{tikz-cd}
\usepackage[colorinlistoftodos]{todonotes}
\usepackage{xparse}
\usepackage{xspace}

\usepackage[T1]{fontenc}
\usepackage[charter]{mathdesign}
\usepackage[scaled]{beramono,berasans}
\usepackage{eucal}
\usepackage{epstopdf}
\usepackage{microtype}
\frenchspacing

\addtolength\parskip{.4ex}
\setlength\parindent{0cm}

\relpenalty=10000
\binoppenalty=10000

% todonotes configuration
\newcounter{todocounter}
\DeclareDocumentCommand\addreference{g}{\stepcounter{todocounter}\todo[color = blue!30, fancyline]{\thetodocounter. Add reference\IfNoValueF{#1}{: #1}}\xspace}
\DeclareDocumentCommand\checkthis{g}{\stepcounter{todocounter}\todo[color = red!50, fancyline]{\thetodocounter. Check this\IfNoValueF{#1}{: #1}}\xspace}
\DeclareDocumentCommand\fixthis{g}{\stepcounter{todocounter}\todo[color = orange!50, fancyline]{\thetodocounter. Fix this\IfNoValueF{#1}{: #1}}\xspace}
\DeclareDocumentCommand\expand{g}{\stepcounter{todocounter}\todo[color = green!50, fancyline]{\thetodocounter. Expand\IfNoValueF{#1}{: #1}}\xspace}
\newcommand\removethis{\stepcounter{todocounter}\todo[color=yellow!50]{\thetodocounter. Remove this?}}

\title{Suggesties voor onderzoekscompetenties}
\author{Pieter Belmans}

\begin{document}
\maketitle

\section{Fouriertransformaties}
In de slides wordt er gebruik gemaakt van fouriertransformaties, zonder deze al te grondig te defini\"eren. Het kan dus interessant zijn om hier verder op te bouwen.\addreference{Matthias: kan jij hier enige referenties voorzien? Jij bent vertrouwder met de literatuur.}
\begin{quote}
  Wat is de geschiedenis van de fouriertransformatie?
\end{quote}
\begin{quote}
  Welke complexe analyse is er nodig om fouriertransformaties te begrijpen?
\end{quote}
\begin{quote}
  Wat is een fouriertransformatie? Wat is een discrete fouriertransformatie? Welke gebruiken we in de analyse van geluid?
\end{quote}
\begin{quote}
  Hoe berekenen we een fouriertransformatie? Hoe wordt een fouriertransformatie in het echt berekend? Wat is FFT?
\end{quote}
\begin{quote}
  Wat zijn nog toepassingen van fouriertransformaties? Denk aan beeldanalyse, patroonherkenning, \dots
\end{quote}


\section{Vergelijking gelijkzwevende en reine stemming}
In de slides wordt er hier kort aandacht aan besteed. Essentieel hierbij is het rekenen met logaritmes (in grondtal 2). Mogelijke vragen die gesteld kunnen worden zijn:
\begin{quote}
  Wat zijn de verschillen voor de andere intervallen?
\end{quote}
\begin{quote}
  Wat gebeurt er bij een verdeling in meer of minder dan 12 delen?
\end{quote}
Deze twee vragen zijn een eenvoudige berekening, gebaseerd op de breuken die bij de reine stemming gebruikt worden, zie \url{http://en.wikipedia.org/wiki/Just_intonation}. De vergelijkende studie die wordt gesuggereerd kan zeer goed dienen om vertrouwd te geraken met wiskundige software:
\begin{quote}
  Hoe kunnen we de verschillen visueel maken?
\end{quote}
\begin{quote}
  Hoe kunnen we de verschillen kwantificeren?
\end{quote}
\begin{quote}
  Kunnen we de verschillen hoorbaar maken?
\end{quote}
Interessant bronmateriaal is te vinden op \url{http://ppexpressivo.co.uk} en \url{http://xenharmonic.wikispaces.com}.


\section{Vergelijking van historische stemmingen}
Verdergaand op de vorige onderzoeksvragen kunnen ook andere stemmingen onderzocht worden. Telkens dienen de volgende vragen beantwoord te worden:
\begin{quote}
  Hoe wordt deze stemming opgebouwd?
\end{quote}
\begin{quote}
  Waarom wordt deze constructie gebruikt?
\end{quote}
\begin{quote}
  Hoe verhoudt deze stemming zich tot de reine en de gelijkzwevende stemming?
\end{quote}
Centraal in de antwoorden op deze vragen staan volgende vragen:
\begin{quote}
  Wat zijn komma's?
\end{quote}
\begin{quote}
  Wat is het wolfsinterval?
\end{quote}
Telkens kan bekeken worden hoe een bepaalde stemming omgaat met deze problemen.

Interessante stemmingen (op basis van historische of wiskundig belang) zijn:
\begin{enumerate}
  \item pythagorische stemming
  \item middentoonsstemmingen (\emph{meantone tuning})
  \item welgetemperde stemmingen (\emph{well-tempered tuning})
\end{enumerate}

Wederom zijn \url{http://ppexpressivo.co.uk} en \url{http://xenharmonic.wikispaces.com} goede bronnen.


\section{\emph{Musica universalis}}
Deze suggestie voor onderzoekscompetenties is enkel van toepassing op klasgroepen die Latijn volgen. De wiskundige inhoud is hier beperkter, het betreft eerder historisch onderzoek.

In de slides wordt volgend tekstfragment vermeld:
\begin{quote}
  Sed Pythagoras interdum et musica ratione appellat quantum absit a terra luna, ab ea ad Mercurium dimidium spatii et ab eo ad Veneris, a quo ad solem sescuplum, a sole ad Martem tonum [id est quantum ad lunam a terra], ab eo ad Iovem dimidium et ab eo ad Saturni, et inde sescuplum ad signiferum; ita septem tonis effici quam \greektext di`a paswn 'armon'ian \latintext hoc est universitatem concentus; in ea Saturnum Dorio moveri phthongo, Iovem Phrygio et in reliquis similia, iucunda magis quam necessaria subtilitate.
\end{quote}
\begin{flushright}
  Plinius de oudere, Naturalis Historia, 77--79 n.C.
\end{flushright}
Een eerste stap is dus
\begin{quote}
  Vertaal en interpreteer dit fragment.
\end{quote}
Daarna zijn er meerdere zaken die onderzocht kunnen worden.
\begin{quote}
  Wat leert dit fragment (aangevuld met ander bronmateriaal ons) over de wereldvisie van Pythagoras?
\end{quote}
\begin{quote}
  Wat zijn nog manifestaties van dit wereldbeeld? (Denk aan: Kepler, Plato, \dots)
\end{quote}
\begin{quote}
  Hoe is dit wereldbeeld ge\"evolueerd tot het huidige wereldbeeld?
\end{quote}
\begin{quote}
  Wat is numerologie?
\end{quote}

\end{document}
